%---------------------------------------------------------------------------%
%->> 封面信息及生成
%---------------------------------------------------------------------------%
%-
%-> 中文封面信息
%-
\confidential{}% 密级:只有涉密论文才填写
\schoollogo{scale=0.5}{nwu_logo}% 校徽
\title{低分辨率环境下的微表情识别}% 论文中文题目
\author{李桂锋}% 论文作者
\advisor{彭进业~教授~西北大学}% 指导教师:姓名 专业技术职务 工作单位
%\advisorsec{}% 指导老师附加信息 或 第二指导老师信息
\degree{硕士}% 学位:学士、硕士、博士
\degreetype{工学}% 学位类别:理学、工学、工程、医学等
\major{电子与通信工程}% 二级学科专业名称
\institute{信息科学与技术学院}% 院系名称
\chinesedate{2019~年~6~月}% 毕业日期:夏季为6月、冬季为12月
%-
%-> 英文封面信息
%-
\englishtitle{Micro-expression Recognition\\Under Low-resolution Case}% 论文英文题目
\englishauthor{Li Guifeng}% 论文作者
\englishadvisor{Supervisor: Professor Peng Jinye}% 指导教师
\englishdegree{Master}% 学位:Bachelor, Master, Doctor。封面格式将根据英文学位名称自动切换,请确保拼写准确无误
\englishdegreetype{Engineering}% 学位类别:Philosophy, Natural Science, Engineering, Economics, Agriculture 等
\englishthesistype{thesis}% 论文类型: thesis, dissertation
\englishmajor{Electronics and Communication Engineering}% 二级学科专业名称
\englishinstitute{College of Information Science and Technology}% 院系名称
\englishdate{June, 2019}% 毕业日期:夏季为June、冬季为December
%-
%-> 生成封面
%-
\maketitle% 生成中文封面
\makeenglishtitle% 生成英文封面
%-
%-> 作者声明
%-
\makedeclaration% 生成声明页
%-
%-> 中文摘要
%-
\chapter*{摘\quad 要}\chaptermark{摘\quad 要}% 摘要标题
\setcounter{page}{1}% 开始页码
\pagenumbering{Roman}% 页码符号

人脸表情在我们的社交互动中发挥着重要作用,因为它传达了丰富的信息。我们可以从一张人脸图像中阅读很多内容,但是如果没有特殊设备,我们也无法感知到这些信息。本文采用计算机视觉方法分析肉眼难以察觉的两种微妙的面部信息:微表情和心率。
微表情是快速、不自主的面部表情,揭示了人们不打算表达的情感。人们很难感知微表情,因为它们太快和微妙,因此自动微表情分析是很有价值的工作,具有重大的应用前景。本文综述了微表情研究的进展,并分四部分工作进行描述。 1)我们介绍了第一个自发的微表情数据库—SMIC。缺乏数据阻碍了微表情的分析研究,因为很难收集自发的微表情。引入用于诱导和注释SMIC的协议以帮助未来的微表情收集。 2)引入了包括三个特征和视频放大过程的框架用于微表情识别,其优于两个微表情数据库上的其他最先进的方法。 3)描述了一种基于特征差异分析的微表情定位方法,该方法可以从自发的长视频中发现为微表情。 4)提出了一种自动微表情分析系统(MESR),用于发现并识别微表情。
心率是我们健康和情绪状态的重要指标。传统的心率测量需要皮肤接触,不能远程应用。我们提出了一种方法,可以对抗照明变化和头部运动,并从彩色面部视频远程测量心率。我们还应用该方法来解决面部反欺骗问题。我们展示了基于脉冲的特征比传统的基于纹理的特征更能够抵抗看不见的掩模欺骗。我们还表明,所提出的基于脉冲的特征可以与其他特征相结合,以构建用于检测多种类型的攻击的级联系统。
最后,我们总结了工作的贡献,并基于当前工作的局限性提出了关于微表情和心率研究的未来计划。还计划将微表情和心率(可能还有来自面部的其他微妙信号)结合起来构建用于情感状态分析的多模式系统。

微表达是一种基本的非言语行为,它能忠实地表达人类隐藏的情感。它在国家安全、计算机辅助诊断等领域有着广泛的应用,促使我们对自动微表情识别进行研究。但从监控视频中获取的图像容易出现质量问题,导致实际应用困难。由于捕获的图像质量较低,现有的算法无法达到预期的效果。为了解决这个问题,我们进行了全面的研究

\keywords{微表情识别,监控视频,低分辨率,超分辨率,Fast LBP-TOP}% 中文关键词
%-
%-> 英文摘要
%-
\chapter*{Abstract}\chaptermark{Abstract}% 摘要标题

The face plays an important role in our social interactions as it conveys rich sources of information. We can read a lot from one face image, but there is also information we cannot perceive without special devices. The thesis concerns using computer vision methodologies to analyse two kinds of subtle facial information that can hardly be perceived by naked eyes: the micro-expression (ME), and the heart rate (HR). 
MEs are rapid, involuntary facial expressions which reveal emotions people do not intend to show. It is difficult for people to perceive MEs as they are too fast and subtle, thus automatic ME analysis is valuable work which may lead to important applications. In the thesis, the progresses of ME studies are reviewed, and four parts of work are described. 1) We introduce the first spontaneous ME database, the SMIC. The lacking of data is hindering ME analysis research, as it is difficult to collect spontaneous MEs. The protocol for inducing and annotating SMIC is introduced to help future ME collections. 2) A framework including three features and a video magnification process is introduced for ME recognition, which outperforms other state-of-the-art methods on two ME databases. 3) An ME spotting method based on feature difference analysis is described, which can spot MEs from spontaneous long videos. 4) An automatic ME analysis system (MESR) was proposed for firstly spotting and then recognising MEs.
The HR is an important indicator of our health and emotional status. Traditional HR measurements require skin-contact which cannot be applied remotely. We propose a method which can counter for illumination changes and head motions and measure HR remotely from color facial videos. We also apply the method for solving the face anti-spoofing problem. We show that the pulse-based feature is more robust than traditional texture-based features against unseen mask spoofs. We also show that the proposed pulse-based feature can be combined with other features to build a cascade system for detecting multiple types of attacks.
At last, we summarize the contributions of the work, and propose future plans about ME and HR studies based on limitations of the current work. It is also planned to combine the ME and HR (maybe also other subtle signals from face) to build a multimodal system for affective status analysis.

Micro-expression is an essential non-verbal behavior that can faithfully express the human's hidden emotions. It has a wide range of applications in the national security and computer aided diagnosis, which encourages us to conduct the research of automatic micro-expression recognition. However, the images captured from surveillance video easily suffer from the low-quality problem, which causes the difficulty in real applications. Due to the low quality of captured images, the existing algorithms are not able to perform as well as expected. For addressing this problem, we conduct a comprehensive study about the micro-expression recognition problem under low-resolution cases with face hallucination method. The experimental results show that the proposed framework obtains promising results on micro-expression recognition under low-resolution cases.

\englishkeywords{Micro-expression recognition, Surveillance video, Low-resolution, Super-resolution, Fast LBP-TOP}% 英文关键词
%---------------------------------------------------------------------------%
